\documentclass[12pt,a4paper]{article}
\usepackage[utf8]{inputenc}
\usepackage[T1]{fontenc}
\usepackage{amsmath}
\usepackage{amsfonts}
\usepackage{amssymb}
\usepackage{multicol}
\usepackage{lmodern}
\usepackage{colortbl}%permet de griser les cases
\usepackage{tabularx, multirow}
\usepackage{lscape}
\usepackage{xcolor}
\usepackage{graphicx}
\usepackage{tikz}
\input{preambulemanu.sty}
\usepackage[left=2cm,right=2cm,top=2cm,bottom=2cm]{geometry}
\def\Oij{$\left(\text{O},~\vec{i},~\vec{j}\right)$}
\usepackage{fancyhdr}


%Permet le code python sur lateX
\usepackage{minted}
\usemintedstyle{lovelace}



\begin{document}
\textbf{2nd} \hfill \textbf{Fiche- Intervalle et valeur absolue } \hfill Lycée Jean Rostand\\
\trait 

\subsection*{Exercice n°1}
 Dans chaque cas, déterminer $\rm I\cap J$ et $\rm I\cup J$:
 
 \begin{enumerate}
     \item $\rm I=[-5;-2[$ et $\rm J=[-4;6[$
     \item  $\rm I=]-\infty;4]$ et $\rm J=[-5;+\infty[$
     \item $\rm I=]6;+\infty[$ et $\rm J=[-4;+\infty[$
 \end{enumerate}
          
\subsection*{Exercice n°2}

 Dans chaque cas, déterminer l'intersection et la réunion des intervalles $\rm I$ et $\rm J$:
 
 \begin{enumerate}
     \item $\rm I=]-4;7]$ et $\rm J=]-\infty;2[$
     \item $\rm I=]-5;3[$ et $\rm J=[3;+\infty[$
     \item $\rm I=]-5;3[$ et $\rm J=]3;+\infty[$

\end{enumerate}
\subsection*{Exercice n°3}

  Dans chaque cas, déterminer $\rm I\cap J$ et $\rm I\cup J$:
  
\begin{enumerate} 
\item $\displaystyle\rm I=\left[\frac 23;\frac 34\right]$ et $\displaystyle\rm J=\left]\frac
                57;\frac 45\right]$
                
\item $\displaystyle\rm I=\left[\frac 23;\frac 45\right]$ et $\displaystyle\rm J=\left]\frac
                57;\frac 34\right]$

\end{enumerate} 

\subsection*{Exercice n°4}

Ecrire les nombres suivants sans valeur absolue:


a) $|-2|$\qquad
b)$|\pi - 3|$\qquad
c) $|\pi -4|$\qquad
d) $|1-\sqrt 2|$\qquad
e)$\displaystyle\left|\frac 2{\sqrt 3}-\sqrt 3\right|$ 


\subsection*{Exercice n°5}
Dans chaque cas, traduire la condition suivante à l'aide d'un intervalle :

a) $|x-1|\leqslant 10^{-2}$ \qquad  b) $|x+2,5|\leqslant 2$

\subsection*{Exercice n°6}
Traduire à l'aide d'une valeur absolue :


a) $y\in [2,4;2,6]$\qquad 
         b) $2\leqslant x \leqslant 7$ \qquad  c) $x\in ]-4;10[$

\subsection*{Exercice n°7}

Résoudre dans $\mathbb{R}$ les équations et inéquations suivantes à l'aide d'un schéma:

 a) $|x+3|=-1$  \qquad b) $|x|>2$ \qquad c)$|x+2|=|1-x|$  \qquad d) $|x-3|\leqslant |x-1|$

\subsection*{Exercice n°8}

   Représenter l'ensemble des points M($x;y$) tels que $
        \left\{
        \begin{array}{rl}
        |x-2| & \leqslant 1 \\
        |y+2| & \leqslant 3
        \end{array}
        \right.$

\end{document}
